\documentclass{aci}

%%%%%%%%%%%%%%%%%%%%%%%%%%%%%%%%%%%%%%%%%%
\usepackage{txfonts}
\usepackage{cite}
\def\typeofarticle{Article type} \def\currentvolume{1} \def\currentissue{1}
\def\currentyear{2021} \def\currentmonth{} \def\ppages{1--30} \def\DOI{to
appear} \def\Received{June 2022} \def\Accepted{July 2022} \def\Published{October
2022 }


\newcommand{\ep}{\varepsilon}
\newcommand{\eps}[1]{{#1}_{\varepsilon}}

%\numberwithin{equation}{section}
\DeclareMathOperator*{\essinf}{ess\,inf}


\begin{document}

\title{Title of the paper}

\author{%
  First name Last name\affil{1}, First name Last name\affil{2} and First name
  Last name\affil{3,}\corrauth}

% \shortauthors is used in copyright information in the end of the paper
\shortauthors{the Author(s)}

\address{%
  \addr{\affilnum{1}}{College of Computing and Software Engineering, Kennesaw
    State University, 1000 Chastain Road, Kennesaw, GA 30144, USA; email}
  \addr{\affilnum{2}}{Affiliation email} \addr{\affilnum{3}}{Affiliation email}}
% corresponding author
\corraddr{Email address of corresponding author; Tel: +1-111-111-1111; Fax:\\
  +1-111-111-1111.}

\editor{ First-name Last-name}

\begin{abstract}
  An abstract is normally of 150 to 250 words. List 4 to 8 keywords all with
  lowercase (except special cases) for both highlighting the focuses of your
  study and indexing/searching in databases. Article type can be one of the
  following: regular article, case study, express letter, review or overview,
  survey, opinion, perspective, communication, lecture notes, tutorial notes,
  and editorial (special issue only).

\end{abstract}

\keywords{(4 to 8 keywords)}



\maketitle

\section{Introduction}

Submission of a manuscript implies that the work described has not been
published before, and not been under consideration for publishing anywhere else.
Your submission also means that the manuscript has been approved by all
co-authors with appropriate permission if required. The publisher will not be
held legally responsible should any dispute be raised for any reason. AIMS-ACI
is an OA publication published quarterly by AIMS Press in February, May, August,
and November. Publishing is currently free of charge. Readers can access all
published articles free of charge through the journal website
https://aimspress.com/journal/aci



\section{Layout and fonts}
The page size is A4 ($8.5 \times 11.0$ in or $21.59 \times 27.94$ cm). The
margin is set to moderate. Font for all text, except symbols and formulae,
should use Times New Roman. Article title is bold with a font size 16 and
aligned to the left. Authors are bold with a font size 12. Using Times New Roman
size 12 for the main text, including figure captions and table titles. However,
text size for contents in a table can be chosen from 8 to 12 to fit the table
within the width of the page.

\section{Headline and sub-headline}
The first-level headline is numbered and bold with a font size 12, and has a
space of 12 $pt$ before and after the headline. Each paragraph begins with an
indentation of 0.63 $cm$ (the same by using the Tab key once) in the first line.
A paragraph does not have a space before and after that paragraph.

\subsection{The second-level headline}
The second-level headline is numbered with a font size 12. The second-level
headline has a default space of 6 $pt$ before and after it.

\subsubsection{The third-level headline (if necessary)}
If really necessary, the third-level headline should be numbered and italicized
with a font size 12, indented by 0.63 $cm$ (the same by using the Tab key once).
The third-level headline has a default space of 6 $pt$ before and after it. Any
lower headline should be avoided.

\subsection{Footnotes}
Although not encouraged, footnotes can be used to provide additional information
for an informal reference that cannot be included in the reference list. A
footnote should be numbered and cannot contain any figure or table. Use font
size 8 for footnotes.

\section{Figures or images, captions, tables, table titles, and formulae}
\subsection{Figure/image and caption}
Leave a space of one line before and after a figure or an image, i.e., one line
between the main text and the top of the figure or image, and one line between
the bottom of the figure or image and the caption. The caption has a default
space of 12 $pt$ after it so the main text can continue below the caption. If no
text follows the figure caption, do not leave any space between the caption and
the next headline (the headline has a default space).

\begin{figure}[ht]
  \centering
  \vskip 0.2in
  %\includegraphics[width=0.85\textwidth]{{logo_new.pdf}}
  \vspace {12pt}
  \caption{ Insert the figure/image here}
  \label{fig:res}
\end{figure}

Figures should be numbered consecutively in the text. A figure can be referred
explicitly in the text as Figure~\ref{fig:res}, or implicitly at the end of a
sentence in parentheses (Fig. 1).

Use black and white graphic for line drawings. All lines should be at least 0.1
mm (0.3 $pt$) wide. Line drawings or scanned line drawings should have a
resolution at least 600 dpi. Images with a large volume should be properly
compressed within 1 MB each. Scanned images should be properly edited to balance
the volume size and clarity of the image.

Figure captions begin with \textbf{Figure x.} in bold, followed by the text. Do
not add a full stop at the end of the caption.


\subsection{Tables and table titles}
\begin{table}[H]
  \begin{center}
    \caption{Publication schedule for AIMS-ACI in 2022}
    \begin{tabular}{lll} \hline
      Issue        & Publishing date   & Submission deadline \\\hline
      First issue  & February 25, 2022 & January 15, 2022    \\
      Second issue & May 25, 2022      & April 15, 2022      \\
      Third issue  & August 25, 2022   & July 15, 2022       \\
      Fourth issue & November 25, 2022 & October 15, 2022    \\\hline
    \end{tabular}
    \label{table}
  \end{center}
\end{table}

Table title should be placed above the table, numbered consecutively, and
referred in the text like Table~\ref{table} or (Table~\ref{table}). There is no
space between the table and table title. Leave a space of one line above the
table title and one line below the bottom of the table. If a new first-level
headline follows the table immediately, do not leave the space below the table.

The font of the table title is size 12 but the font of the table contents can be
sizes 8 to 12 depending on the best fit for the table. Table title begins with
\textbf{Table x}. in bold, followed by the text. Do not add a full stop at the
end of the table title.




\subsection{Mathematical formulae}
Mathematical formulae should be numbered consecutively by placing the number in
parentheses like (1), one TAB space away from the right line. Mathematical
formulae should be typed using MS Word Equation or better using \emph{MathType}.
Use size 10--12 for the main characters and size 6--8 for subscripts or
superscripts. Leave a space of 6 $pt$ between two formula in separate lines.
Some examples are shown below.

\begin{equation}
  \lim_{x \mapsto a^+ }f\left ( x \right )= \pm \infty \qquad \text{or} \qquad \lim_{x \mapsto a^- }f\left ( x \right )= \pm \infty
  \label{equation1}
\end{equation}

\begin{equation}
  \lim_{x \mapsto \infty }f\left ( x \right )=  a \qquad \text{or} \qquad \lim_{x \mapsto -\infty}f\left ( x \right )=  b
  \label{equation2}
\end{equation}

To cite a formula in the text, use the number like Equation~(\ref{equation1}),
or Eq.~(\ref{equation2}), or alike.

\section{Submission, review, and decision}
In the first year, submission, review and decisions of manuscripts is conducted
through email communications. Please send your manuscript to one of the relevant
Editors-in-Chief by their email available on the journal Website. With the
increase in submissions from 2022, we would use the AIMS Editorial System to
handle the process of submission, review, and decisions.

Once a submission is received, the receiving Editor will pre-screen the article
for its suitability and quality for the journal. After passing this
pre-screening, the Editor may act as the “Academic Editor” or assign this
manuscript to another Editor from the Editorial Board as the “Academic Editor”
to handle the review of this paper, communicate with the authors on revisions,
and inform the final decision after consultation with the relevant
Editor-in-Chief.

AIMS-ACI holds a high standard in peer review. A submitted manuscript should be
reviewed by at least three independent reviewers who are scholars and
knowledgeable of the area of research. Standard review forms are available for
reviewers but they could choose own appropriate way of reporting to the Academic
Editor.


\section{Reference style, citation, and cross-reference}
Test citation \cite{Jensen1996}.


\section*{Acknowledgments}
We would like to thank the constructive feedback provided by the reviewers.



\bibliographystyle{unsrt}
\bibliography{paper}


\end{document}
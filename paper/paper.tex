\documentclass{aci}

%%%%%%%%%%%%%%%%%%%%%%%%%%%%%%%%%%%%%%%%%%
\usepackage{txfonts}
\usepackage{cite}
\usepackage{booktabs}

\usepackage{hyperref}
\hypersetup{colorlinks=true}

%%%%%%%%%%%%%%%%%%%%%%%%%%%%%%%%%%%%%%%%%%
\newcommand{\ep}{\varepsilon}
\newcommand{\eps}[1]{{#1}_{\varepsilon}}

\def\typeofarticle{Research Article} \def\currentvolume{1} \def\currentissue{1}
\def\currentyear{2021} \def\currentmonth{} \def\ppages{1--30} \def\DOI{to
appear} \def\Received{June 2022} \def\Accepted{July 2022} \def\Published{October
2022 }
%\numberwithin{equation}{section}
\DeclareMathOperator*{\essinf}{ess\,inf}

%%%%%%%%%%%%%%%%%%%%%%%%%%%%%%%%%%%%%%%%%%
\begin{document}

\title{An Ant Cuticle Texture Classification Algorithm for Ecological Anaylsis}

\author{%
  Noah Gardner\affil{1}, John Paul Hellenbrand\affil{2}, and Chih-Cheng
  Hung\affil{1}\corrauth}

% \shortauthors is used in copyright information in the end of the paper
\shortauthors{the Author(s)}

\address{%
  \addr{\affilnum{1}}{College of Computing and Software Engineering, Kennesaw
    State University, 1000 Chastain Road, Kennesaw, GA 30144, USA; email}
  \addr{\affilnum{2}}{College of Science and Mathematics, Kennesaw State
    University, 1000 Chastain Road, Kennesaw, GA 30144, USA; email}}
% corresponding author
\corraddr{chung1@kennesaw.edu}

\editor{First-name Last-name}

\begin{abstract}
  There is a large variety of ant species, and most species are diverse in terms
  of size, shape, behaviors, and especially skin (cuticle) textures. However,
  the significance of ant cuticle texture is not widely researched. This
  research employs modern machine learning methods such as texture analysis and
  classification with CNN and clustering to automatically group similar ant
  species to allow for the study of influences cuticle texture on ant ecology.

\end{abstract}

\keywords{Texture Analysis, Image Processing, Clustering, Machine Learning,
  Myrmecology, Ecology}



\maketitle

\section{Introduction}
Insects compose half of biodiversity and rank among the most dominant organisms
in terrestrial ecosystems \cite{sheikh_diverse_2017}. A key factor for the
ecological success of insects is their exoskeleton, also known as cuticle. The
cuticle protects insects from predation, provides structural support, prevents
desiccation, and serves as a canvas for advertising visual and chemical signals
\cite{gullan_insects_2009}. Research has heavily focused on the macrostructures
and internal chemical components that make the exoskeleton functional and more
recent work is being done to understand the functional aspects of external
cuticle micro sculpturing \cite{muthukrishnan_insect_2020,
  gunderson_insect_1989, watson_diversity_2017}.

We examine ants (\textit{Formicidae}) as they display an extreme diversity of
cuticle micro sculpturing across all subfamilies. Sculpturing ranges from
parallel longitudinal ridges to deep oval impressions to erratic protuberances.
The sculpturing has arisen convergently and independently throughout ant’s
evolutionary history, which suggests some inherent function. Cuticle sculpturing
on ants may help increase strength and rigidness, resist abrasion, increase
internal and external surface area, resist microbial growth, and rear beneficial
anti-biotic producing bacteria \cite{johnson_effect_2011,
  bruckner_relationship_2017, currie_coevolved_2006}. These specific functions may
be associated with certain sculpturing types and the purpose of classification
is to group similar textures based on proposed function.

\section{Methods}

\subsection{Sculpture Identification Protocol}

\subsection{Dataset}
In order to classify the cuticle, we used ant head images sourced from AntWeb
\cite{antweb}. In general, the ant head images are centered in the image, facing
the front, and share a similar posture. However, some images may not be
centered, show the ant head in a different orientation, or may have a
drastically different resolution from the average image. Additionally, images
can vary in scale. Fortunately, each can have several scales available such that
the images' scales can be roughly similar.

\begin{table}
  \centering
  \begin{tabular}{ll}
\toprule
{} & Samples \\
\midrule
Rough Dimpled  &     173 \\
Rough Netted   &     503 \\
Rough Ridged   &     317 \\
Rough Tuberous &      41 \\
Smooth Gritty  &      16 \\
Smooth         &    1393 \\
Total          &    2443 \\
\bottomrule
\end{tabular}

\end{table}

Images were collected and classified by undergraduate students according to the
sculpturing identification protocol. Overall, there are a total of 2443 images
in the dataset. The class spread is shown in

\section{Experimental Results}


\section*{Acknowledgments}
We would like to thank the constructive feedback provided by the reviewers.

\bibliography{paper}
\bibliographystyle{unsrt}

\end{document}
